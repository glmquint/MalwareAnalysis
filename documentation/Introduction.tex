\chapter{Introduction}
The aim of this project is to study and analyze the behavior of samples of viruses for Android devices.

The files received were 5, one was a zip file containing an APK and the remaining 4 were without an extension. Following the list of the files:
\begin{itemize}
	\item 0b8bae30da84fb181a9ac2b1dbf77eddc5728fab8dc5db44c11069fef1821ae6
	\item 0b41181a6b9c85b8fa5c8e8c836ac24dd6e738a0d843f0b81b46ffe41b925818
	\item 0c05e5035951e260725d15392c8792a4941f92f868558e8b90b52977d832a70d
	\item 0c40fb505fb96ca9aed220f48a3c6c22318d889efa62bc7aaeee98f3a740afab
	\item 355cd2b71db971dfb0fac1fc391eb4079e2b090025ca2cdc83d4a22a0ed8f082.zip
\end{itemize}

The files are name accordingly to the output of the SHA256 of their digest. After a first analysis it was discovered that the fist 4 files contained malwere very similar between them, meanwhile the last one was different (this will be explained in the next chapters).
Before moving on we shall discuss the tools used for our analysis.
\begin{itemize}
	\item VirusTotal (\href{www.virustotal.com}{www.virustotal.com}): this online tool allows to submit APK samples and analyze them with several anti-virus and anti-malware programs. This tool was used to gain precious insight on the malicious software
	\item MobSF (\href{mobsf.live}{mobsf.live}): This tool is used for both static and dynamic analysis. This allows user to gain stronger insight on how the software works and how the permission are used.
	\item Bytecode Viewer: This tool allows users to decompile bytecode into java code, allowing analyst to scrutinize it.
\end{itemize}

