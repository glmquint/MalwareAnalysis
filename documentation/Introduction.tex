\chapter{Introduction}
The aim of this project is to study and analyze the behavior of samples of viruses for Android devices.

The files received were 5, one was a zip file containing an APK and the remaining 4 were without an extension. Following the list of the files:
\begin{itemize}
	\item \texttt{0b8bae30da84fb181a9ac2b1dbf77eddc5728fab8dc5db44c11069fef1821ae6}
	\item \texttt{0b41181a6b9c85b8fa5c8e8c836ac24dd6e738a0d843f0b81b46ffe41b925818}
	\item \texttt{0c05e5035951e260725d15392c8792a4941f92f868558e8b90b52977d832a70d}
	\item \texttt{0c40fb505fb96ca9aed220f48a3c6c22318d889efa62bc7aaeee98f3a740afab}
	\item \texttt{355cd2b71db971dfb0fac1fc391eb4079e2b090025ca2cdc83d4a22a0ed8f082.zip}
\end{itemize}

The files are name accordingly to their SHA256 digest. After a first analysis it was discovered that the fist 4 files contained malware very similar between them, meanwhile the last one was different (this will be explained in the next chapters).
Before moving on, we shall discuss the tools used for our analysis.
\begin{itemize}
	\item VirusTotal (\url{https://www.virustotal.com}): this online tool allows to submit APK samples and analyze them with several anti-virus and anti-malware programs. This tool was used to gain precious insights on the malicious software.
	\item MobSF (\url{https://mobsf.live}): This tool is used for performing both static and dynamic analysis. This allows users to gain deeper insights on how the software works and how the permissions are used.
	\item Bytecode Viewer (\url{https://www.bytecodeviewer.com/}): This tool allows users to disassemble bytecode into a plausible java source code, allowing analysts to scrutinize it.
	\item apktool (\url{https://apktool.org/}): This tool allows to decode the apk and obtain the application resources like images, layout templating files and the Java bytecode. It can also be used to invert this process, and build an apk from the resulting resource files.
\end{itemize}

